% standard macros for WEB listings (in addition to PLAIN.TEX)
\xdef\fmtversion{\fmtversion+WEBMAC4.1} % identifies current set of macros
\parskip 0pt % no stretch between paragraphs
\parindent 1em % for paragraphs and for the first line of Pascal text

\font\eightrm=cmr8 \let\sc=\eightrm % NOT a caps-and-small-caps font!
\let\mainfont=\tenrm
\font\titlefont=cmr7 scaled\magstep4 % title on the contents page
\font\ttitlefont=cmtt10 scaled\magstep2 % typewriter type in title
\font\tentex=cmtex10 % TeX extended character set (used in strings)
\fontdimen7\tentex=0pt % no extra space after punctuation

\def\\#1{\hbox{\it#1\/\kern.05em}} % italic type for identifiers
\def\|#1{\hbox{$#1$}} % one-letter identifiers look a bit better this way
\def\&#1{\hbox{\bf#1\/}} % boldface type for reserved words
\def\.#1{\hbox{\tentex % typewriter type for strings
  \let\\=\BS % backslash in a string
  \let\'=\RQ % right quote in a string
  \let\`=\LQ % left quote in a string
  \let\{=\LB % left brace in a string
  \let\}=\RB % right brace in a string
  \let\~=\TL % tilde in a string
  \let\ =\SP % space in a string
  \let\_=\UL % underline in a string
  \let\&=\AM % ampersand in a string
  #1}}
\def\#{\hbox{\tt\char`\#}} % parameter sign
\def\${\hbox{\tt\char`\$}} % dollar sign
\def\%{\hbox{\tt\char`\%}} % percent sign
\def\^{\ifmmode\mathchar"222 \else\char`^ \fi} % pointer or hat
% circumflex accents can be obtained from \^^D instead of \^
\def\AT!{@} % at sign for control text

\chardef\AM=`\& % ampersand character in a string
\chardef\BS=`\\ % backslash in a string
\chardef\LB=`\{ % left brace in a string
\def\LQ{{\tt\char'22}} % left quote in a string
\chardef\RB=`\} % right brace in a string
\def\RQ{{\tt\char'23}} % right quote in a string
\def\SP{{\tt\char`\ }} % (visible) space in a string
\chardef\TL=`\~ % tilde in a string
\chardef\UL=`\_ % underline character in a string

\newbox\bak \setbox\bak=\hbox to -1em{} % backspace one em
\newbox\bakk\setbox\bakk=\hbox to -2em{} % backspace two ems

\newcount\ind % current indentation in ems
\def\1{\global\advance\ind by1\hangindent\ind em} % indent one more notch
\def\2{\global\advance\ind by-1} % indent one less notch
\def\3#1{\hfil\penalty#10\hfilneg} % optional break within a statement
\def\4{\copy\bak} % backspace one notch
\def\5{\hfil\penalty-1\hfilneg\kern2.5em\copy\bakk\ignorespaces}% optional break
\def\6{\ifmmode\else\par % forced break
  \hangindent\ind em\noindent\kern\ind em\copy\bakk\ignorespaces\fi}
\def\7{\Y\6} % forced break and a little extra space

\def\to{\mathrel{.\,.}} % double dot, used only in math mode
\def\note#1#2.{\Y\noindent{\hangindent2em\baselineskip10pt\eightrm#1~#2.\par}}
\def\lapstar{\rlap{*}}
\def\startsection{\noindent{\let\*=\lapstar\bf\modstar.\quad}}
\def\defin#1{\global\advance\ind by 2 \1\&{#1 }} % begin `define' or `format'
\def\A{\note{See also section}} % crossref for doubly defined section name
\def\As{\note{See also sections}} % crossref for multiply defined section name
\def\C#1{\ifmmode\gdef\XX{\null$\null}\else\gdef\XX{}\fi % Pascal comments
  \XX\hfil\penalty-1\hfilneg\quad$\{\,$#1$\,\}$\XX}
\def\D{\defin{define}} % macro definition
\def\E{\cdot10^} % exponent in floating point constant
\def\ET{ and~} % conjunction between two section numbers
\def\ETs{, and~} % conjunction between the last two of several section numbers
\def\F{\defin{format}} % format definition
\let\G=\ge % greater than or equal sign
\def\H#1{\hbox{\rm\char"7D\tt#1}} % hexadecimal constant
\let\I=\ne % unequal sign
\let\K=\gets % left arrow
\let\L=\le % less than or equal sign
\outer\def\M#1.{\MN#1.\ifon\vfil\penalty-100\vfilneg % beginning of section
  \vskip12ptminus3pt\startsection\ignorespaces}
\outer\def\N#1.#2.{\MN#1.\vfil\eject % beginning of starred section
  \def\rhead{\uppercase{\ignorespaces#2}} % define running headline
  \message{*\modno} % progress report
  \edef\next{\write\cont{\Z{#2}{\modno}{\the\pageno}}}\next % to contents file
  \ifon\startsection{\bf\ignorespaces#2.\quad}\ignorespaces}
\def\MN#1.{\par % common code for \M, \N
  {\xdef\modstar{#1}\let\*=\empty\xdef\modno{#1}}
  \ifx\modno\modstar \onmaybe \else\ontrue \fi \mark{\modno}}
\def\O#1{\hbox{\rm\char'23\kern-.2em\it#1\/\kern.05em}} % octal constant


\let\R=\lnot % logical not
\let\S=\equiv % equivalence sign
\let\V=\lor % logical or
\let\W=\land % logical and
\def\X#1:#2\X{\ifmmode\gdef\XX{\null$\null}\else\gdef\XX{}\fi % section name
  \XX$\langle\,$#2{\eightrm\kern.5em#1}$\,\rangle$\XX}
\let\Z=\let % now you can \send the control sequence \Z
\let\~=\ignorespaces
\let\*=*

\def\onmaybe{\let\ifon=\maybe} \let\maybe=\iftrue
\newif\ifon \newif\iftitle \newif\ifpagesaved
\def\lheader{\mainfont\the\pageno\eightrm\qquad\rhead\hfill\title\qquad
  \tensy x\mainfont\topmark} % top line on left-hand pages
\def\rheader{\tensy x\mainfont\topmark\eightrm\qquad\title\hfill\rhead
  \qquad\mainfont\the\pageno} % top line on right-hand pages
\def\page{\box255 }
\def\normaloutput#1#2#3{\ifodd\pageno\hoffset=\pageshift\fi
  \shipout\vbox{
    \vbox to\fullpageheight{
      \iftitle\global\titlefalse
      \else\hbox to\pagewidth{\vbox to10pt{}\ifodd\pageno #3\else#2\fi}\fi
      \vfill#1}} % parameter #1 is the page itself
  \global\advance\pageno by1}

\def\rhead{\.{WEB} OUTPUT} % this running head is reset by starred sections
\def\title{} % an optional title can be set by the user
\def\topofcontents{\centerline{\titlefont\title}
  \vfill} % this material will start the table of contents page
\def\botofcontents{\vfill} % this material will end the table of contents page
\def\contentspagenumber{0} % default page number for table of contents
\newdimen\pagewidth \pagewidth=6.5in % the width of each page
\newdimen\pageheight \pageheight=8.7in % the height of each page
\newdimen\fullpageheight \fullpageheight=9in % page height including headlines
\newdimen\pageshift \pageshift=0in % shift righthand pages wrt lefthand ones
\def\magnify#1{\mag=#1\pagewidth=6.5truein\pageheight=8.7truein
  \fullpageheight=9truein\setpage}
\def\setpage{\hsize\pagewidth\vsize\pageheight} % use after changing page size

\output{\setbox0=\page % the first page is garbage
  \global\output{\normaloutput\page\lheader\rheader}}
\setpage
\vbox to \vsize{} % the first \topmark won't be null

% Go into Pascal mode
\rightskip=0pt plus 100pt minus 10pt % go into Pascal mode
\sfcode`;=3000
\pretolerance 10000
\hyphenpenalty 10000 \exhyphenpenalty 10000
\global\ind=2 \1\ \unskip

\tracingoutput=1
\showboxbreadth=\maxdimen
\showboxdepth=\maxdimen
\tracingonline=1

$\4\X16:Make the first 256 strings\X\S$\6
\&{for} $\|k\K0\mathrel{\&{to}}255$ \1\&{do}\6
\&{begin} \37$\\{write}(\|k:3,\39\.{\':\ "\'})$;\5
$\|l\K\|k$;\6
\&{if} $(\X17:Character \|k cannot be printed\X)$ \1\&{then}\6
\&{begin} \37$\\{write}(\\{xchr}[\.{"\^"}],\39\\{xchr}[\.{"\^"}])$;\6
\&{if} $\|k<\O{100}$ \1\&{then}\5
$\|l\K\|k+\O{100}$\6
\4\&{else} \&{if} $\|k<\O{200}$ \1\&{then}\5
$\|l\K\|k-\O{100}$\6
\4\&{else} \&{begin} \37$\\{lc\_hex}(\|k\mathbin{\&{div}}16)$;\5
$\\{write}(\\{xchr}[\|l])$;\5
$\\{lc\_hex}(\|k\mathbin{\&{mod}}16)$;\5
$\\{incr}(\\{count})$;\6
\&{end};\2\2\6
$\\{count}\K\\{count}+2$;\6
\&{end};\2\6
\&{if} $\|l=\.{""}\.{""}$ \1\&{then}\5
$\\{write}(\\{xchr}[\|l],\39\\{xchr}[\|l])$\6
\4\&{else} $\\{write}(\\{xchr}[\|l])$;\2\6
$\\{incr}(\\{count})$;\5
$\\{write\_ln}(\.{\'"\'})$;\6
\&{end}\2\par



\bye